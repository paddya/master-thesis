\chapter{Preliminaries}
\label{ch:preliminaries}

In this thesis, $G = (V, E)$ denotes a simple, unweighted graph with a set of nodes $V$ and a set of edges $E$ between the nodes. An undirected edge is a subset of $V$ with exactly two distinct elements. A directed edge is an element of $V \times V$. The distance between two nodes $u$ and $v$ is denoted by $d(u, v)$. There are sometimes multiple graph instances $G$ and $G'$ when dealing with dynamic graphs. In these cases, we will use $d_G(u, v)$ to refer to distances in $G$.

The set of neighbor nodes for $v$ is denoted by $N(v) := \{u : \exists (v, u) \in E\}$ in a directed graph and by $N(v) := \{u : \exists \{v, u\} \in E\}$ in an undirected graph. It is sometimes useful to consider the incoming edges of a node in directed graphs. $N^{\leftarrow}(v) := \{u : \exists (u, v) \in E\}$ denotes the set of incoming neighbors of a node $v$.

We now want to define the concept of closeness centrality. The most simple definition, while only meaningful for connected graphs, is the following:

\begin{definition}
\label{def:closenessConnected}
Let $G = (V, E)$ be a connected, unweighted graph. The closeness centrality of a node $v \in V$ is defined as
\begin{align*}
	c(v) = \frac{|V| - 1}{\sum_{u \in V}{d(v, u)}}.
\end{align*}
\end{definition}

In the disconnected case, there are node pairs without a path between them. Using Definition~\ref{def:closenessConnected} and assuming $d(u, v) = \infty$ for such node pairs, the sum over all the distances in the denominator would blow up and make the resulting closeness centralities useless. To solve this problem, we first define $R(v) \coloneqq \{u \in V : u \text{ is reachable from }v \text{ in } G\}$ and $r(v) \coloneqq |R(v)|$. This leads to a generalized version of Definition~\ref{def:closenessConnected}:
\begin{align}
	c(v) = \frac{r(v) - 1}{\sum_{u \in R(v)}{d(v, u)}}. \label{eq:closenessIncomplete}
\end{align}
However, this definition does not differentiate between central nodes in small components and central nodes in large components of a graph. Intuitively, a node $v$ with $r(v) = 2000$ and a total distance of $4000$ to all reachable nodes is more central than a node $w$ with $r(w) = 20$ and a total distance of $40$. In order to give preference to nodes in large components, we scale Equation~\ref{eq:closenessIncomplete} by $\frac{r(v)}{|V| - 1}$. 

\begin{definition}
\label{def:closenessGeneral}
Let $G = (V, E)$ be an unweighted graph. The closeness centrality of a node $v \in V$ is defined as
\begin{align*}
	c(v) = \frac{r(v) - 1}{\sum_{u \in R(v)}{d(v, u)}} \cdot \frac{r(v)}{n - 1}.
\end{align*}
\end{definition}

Another approach to handle disconnected graphs is called \emph{harmonic closeness centrality}.

\begin{definition}
\label{def:harmonicCentrality}
Let $G = (V, E)$ be an unweighted graph. The harmonic closeness centrality of a node $v \in V$ is defined as
\begin{align*}
	h(v) = \sum_{u \in V}{\frac{1}{d(v, u)}}\text{~\cite{boldi2014axioms}}.
\end{align*}
\end{definition}
If there is no path between $u$ and $v$, the inverse distance is set to $0$. Unless stated otherwise, \emph{closeness centrality} refers to \emph{harmonic closeness centrality} in this thesis.