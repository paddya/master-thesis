\chapter{Group closeness}
\label{ch:groupCloseness}

The concept of closeness centrality can be extended to groups of nodes. Sometimes, an application does not require a list of important nodes, but rather a group of nodes that in combination has a large influence. The most important nodes of a graph are often close to each other, so their ``spheres of influence'' overlap. In other words, the most important nodes might have similar distances to most nodes in the graph. In a group of the most important nodes, the individual contribution of some nodes might therefore be minimal. Instead, it might make more sense to ``spread out'' the nodes of the group over the whole graph in order to minimize the distances to as many nodes as possible.

Chen et al. call this problem \emph{Maximum Closeness Centrality Group Identification} (MCGI) in~\cite{Chen2016}. They show the problem to be NP-hard.

\section{Problem definition}

Let $G = (V, E)$ denote an undirected connected graph. Let $S \subseteq V$ denote a group of $k$ nodes. We define the distance of a node $v$ to the group as $d_S(v) := min_{s \in S}d(s, v)$. The \emph{group closeness} of $S$ is defined as
\begin{align}
	c(S) = \frac{|V|}{\sum_{v \in V \setminus S}{d_S(v)}}.
\end{align}
The goal of the Maximum Closeness Centrality Group Identification problem is to find the group $S^*$ which maximizes $c(S^*)$ among all groups $S^*$ of size $k$.
Figure~\ref{fig:groupCloseness} illustrates the problem. The yellow nodes belong to a group of size $3$.

\begin{figure}[h!]
\centering
\begin{tikzpicture}[scale=0.8]
    \tikzstyle group node=[main node,fill=yellow!80];
    
	\node[group node] (v) {};
    \foreach \a in {1,2,...,6}{
		\draw (\a*360/6: 2cm) node[main node] (v\a) {};
		\draw (v) to (v\a);
	}
	
	\begin{scope}[xshift=6cm]
	
	\node[group node] (u) {};
    \foreach \a in {1,2,...,6}{
		\draw (\a*360/6: 2cm) node[main node] (u\a) {};
		\draw (u) to (u\a);
	}
	\end{scope}
	
	
	\begin{scope}[yshift=-6cm, xshift=3cm]
	
	\node[group node] (w) {};
    \foreach \a in {1,2,...,6}{
		\draw (\a*360/6: 2cm) node[main node] (w\a) {};
		\draw (w) to (w\a);
	}
	\end{scope}
	
	\draw (u) to [bend right] (v);
	\draw (u) to [bend left] (w);
	\draw (v) to [bend right] (w);
	
	\draw (v4) to (w3);
	\draw (u5) to [bend left] (w5);
\end{tikzpicture}
\caption{Group closeness}{}
\label{fig:groupCloseness}
\end{figure}

\section{Computing group closeness}
The group closeness function $c(S)$ has two properties that can be exploited to design an approximative algorithm for the problem: monotonicity and submodularity.

\begin{itemize}
	\item Monotonicity: Given any two sets $S, T \subseteq V$ with $S \subseteq T$, a function $f(x)$ is monotonic, if $f(S) \leq f(T)$.
	\item Submodularity: Given any two sets $S, T \subseteq V$ with $S \subseteq T$ and $v \in V \setminus T$, a function $f(x)$ is submodular, if $f(S \cup \{v\}) - f(S) \geq f(T \cup \{v\}) - f(T)$.
\end{itemize}
Chen et al. provide a proof that the group closeness function has these properties in~\cite{Chen2016}.

\subsection{Greedy algorithm by Chen et al.}
Since the group closeness function is submodular and monotonic, there exists a greedy algorithm that computes a set $S$ with $k$ nodes that is a $1 - \frac{1}{e}$ approximation of the optimal solution $S^*$. It has a space complexity of $\mathcal{O}(n^2)$ and a time complexity of $\mathcal{O}(n^3)$.

Algorithm~\ref{alg:chenGroupCloseness} outlines the greedy algorithm. It maintains a set $S$ which contains the nodes already selected in previous iterations. The matrix $M$ stores the distance $d_{S \cup \{u\}}(v)$ ($v \in V$) for each node $u$. For instance, if a node $u$ has distance $4$ to a node $w$ and the group $S$ has distance $5$ to $w$, $M[u, w]$ would store $4$. $Score$ is a vector that stores the group closeness if a node $u$ is added to $S$, i.e. $c(S \cup \{u\})$. Initially each entry in $Score$ is set to the closeness centrality of the corresponding node (Line~\ref{alg:initScore}). In the main loop (Line~\ref{alg:chenGroupClosenessLoop}), the algorithm selects the node with the highest value in $Score$ in each iteration for the group $S$ (Line~\ref{alg:selectNode}). After the node is selected, the distances stored in $M$ are updated to reflect any new shortest paths between the group $S$ and other nodes in the graph (Line~\ref{alg:groupDistanceUpdate}). At last, the $Score[u]$ is updated for each node $u \in V \setminus S$ (Line~\ref{alg:groupClosenessScoreUpdate}).

\begin{algorithm2e}[h!]
  \label{alg:chenGroupCloseness}
  \KwData{$G = (V, E)$: a social network, $k$: a positive integer}
  \KwResult{$S$: a set of $k$ nodes}
 
  $S \gets \emptyset$ \\
  $M \gets $ all shortest path distances \\
  $Score \gets \{c(u) \mid u \in V\}$ \label{alg:initScore} \\
  \While{$|S| < k$}{ \label{alg:chenGroupClosenessLoop}
    $v \gets arg\,max_{w \in V \setminus S}{Score[w]}$ \label{alg:selectNode} \\
    $S \gets S \cup \{v\}$ \\
    \ForEach{$u \in V \setminus S$}{
      \ForEach{$w \in V$}{
        \If{$d(u, w) > d(v, w)$}{ \label{alg:groupDistanceUpdate}
          $M[u, w] = d(v, w)$
        }
      }
      $Score[u] \gets c(S \cup \{u\})$ \label{alg:groupClosenessScoreUpdate}
    }
  
  }
 
  \caption{Greedy algorithm to approximate the group with maximum group closeness}
\end{algorithm2e}

\subsection{Improvements by Bergamini and Meyerhenke}

