\chapter{Group closeness}
\label{ch:groupCloseness}

The concept of closeness centrality can be extended to groups of nodes. Sometimes, an application does not require a list of important nodes, but rather a group of nodes that in combination has a large influence. The most important nodes of a graph are often close to each other, so their ``spheres of influence'' overlap. In other words, the most important nodes might have similar distances to most nodes in the graph. In a group of important nodes, the individual contribution of some nodes might be minimal.



\section{Preliminaries}
Let $S \subseteq V$ denote a group of $k$ nodes. We define the distance of a node $v$ to the group as $d_S(v) := min_{s \in S}d(s, v)$. 

\begin{figure}[h!]
\centering
\begin{tikzpicture}
    \tikzstyle group node=[main node,fill=yellow!80];
    
	\node[group node] (v) {};
    \foreach \a in {1,2,...,6}{
		\draw (\a*360/6: 2cm) node[main node] (v\a) {\a};
		\draw (v) to (v\a);
	}
	
	\begin{scope}[xshift=6cm]
	
	\node[group node] (u) {};
    \foreach \a in {1,2,...,6}{
		\draw (\a*360/6: 2cm) node[main node] (u\a) {\a};
		\draw (u) to (u\a);
	}
	\end{scope}
	
	
	\begin{scope}[yshift=-6cm, xshift=3cm]
	
	\node[group node] (w) {};
    \foreach \a in {1,2,...,6}{
		\draw (\a*360/6: 2cm) node[main node] (w\a) {\a};
		\draw (w) to (w\a);
	}
	\end{scope}
	
	\draw (u) to [bend right] (v);
	\draw (u) to [bend left] (w);
	\draw (v) to [bend right] (w);
	
	\draw (v4) to (w3);
	\draw (u5) to [bend left] (w5);
\end{tikzpicture}
\caption{Group closeness}{}
\label{fig:groupCloseness}
\end{figure}