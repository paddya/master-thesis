\documentclass{thesisclass}

%% -------------------------------
%% |    IM Thesis Template       |
%% -------------------------------
%% Further additions by: Philipp Stroehle, IM, 2013 
%% philipp.stroehle "at" kit.edu

%% Notes:
%% Language switch after \begin{document}

% Based on thesisclass.cls of Timo Rohrberg, 2009
% ----------------------------------------------------------------
% Thesis - Main document
% ----------------------------------------------------------------

% Select your thesis language here:
\usepackage[english]{babel}
%\usepackage[ngerman]{babel}


%% ---------------------------------
%% |      Additional packages      |
%% ---------------------------------
%% 

\usepackage{graphicx}
%http://en.wikibooks.org/wiki/LaTeX/Importing_Graphics#Graphics_storage
\DeclareGraphicsExtensions{.pdf,.png,.jpg}
\graphicspath{{./figures/}} %Use curly braces for each path to add and don't
% forget trailing slash '/'
% \usepackage{epstopdf} %Nice to automatically convert eps figures to pdf
% format  (from inkscape, etc)
\usepackage[square,sort,comma,numbers]{natbib}
\usepackage{booktabs}
\usepackage{amsmath}
\usepackage{amsthm}
\usepackage{amsfonts}
\usepackage{amssymb}
\usepackage{mathtools}
\usepackage{tikz}
\usepackage{booktabs}
\usepackage{multirow}
\usepackage{float}
\usetikzlibrary{positioning}
\usetikzlibrary{calc}
\usepackage[algo2e,linesnumbered]{algorithm2e}
\usepackage{pdflscape} 
\usepackage{placeins}
\usepackage[procnames]{listings}
\usepackage{color}
\usepackage[draft]{todonotes}
\usepackage{csquotes}
	
\usetikzlibrary{positioning}
\tikzset{main node/.style={circle,fill=blue!20,draw,minimum size=1cm,inner sep=0pt},}

\usetikzlibrary{arrows}


%% ---------------------------------
%% | Needed for the List of Abbreviations |
%% ---------------------------------
\usepackage{nomencl}
\renewcommand{\nomname}{List of Abbreviations}
% Punkte zw. Abkürzung und Erklärung
\setlength{\nomlabelwidth}{.32\hsize}
%.32\hsize bezieht sich auf die Punkte
% zwischen Abkürzung und Erklärung -> je größer die Zahl, desto mehr
% Punkte
\renewcommand{\nomlabel}[1]{#1 \dotfill}
% Zeilenabstäde verkleinern
\setlength{\nomitemsep}{-\parsep}
\makenomenclature
\usepackage[normalem]{ulem}
\newcommand{\markup}[1]{\uline{#1}}

%% ---------------------------------
%% |      Theorems and definitions |
%% ---------------------------------
%% 

\newtheorem{definition}{Definition}[chapter]
\newtheorem{proposition}{Proposition}[chapter]
%% ---------------------------------
%% | Information about the thesis  |
%% ---------------------------------

% uncomment one of the following, according to your thesis
\newcommand{\mytype}{\iflanguage{english}{Master's Thesis}{Masterarbeit}} 
%\newcommand{\mytype}{\iflanguage{english}{Master's Thesis}{Masterarbeit}} 
%\newcommand{\mytype}{\iflanguage{english}{Seminar Thesis}{Seminararbeit}} 

\newcommand{\myname}{Patrick Bisenius}
\newcommand{\matricle}{1640015}
\newcommand{\mytitle}{\iflanguage{english}{Computing Top-k
Closeness Centrality in Fully-dynamic Graphs}{Berechnung von
Top-k-Nähezentralität in volldynamischen Graphen}}
\newcommand{\myinstitute}{\iflanguage{english}
{At the Department of Informatics\\Institute of Theoretical Informatics}
{Institut für Theoretische Informatik (ITI)}
}

\newcommand{\reviewerone}{Jun. Prof. Dr. Henning Meyerhenke}
\newcommand{\reviewertwo}{?}
\newcommand{\advisor}{Elisabetta Bergamini}
\newcommand{\advisortwo}{}

\newcommand{\timestart}{\iflanguage{english}{1st December 2016}{1. Januar
2016}}
\newcommand{\timeend}{\iflanguage{english}{31st May 2017}{31. Mai 2017}}
\newcommand{\submissiontime}{02. Mai 2017}

%% -------------------------------
%% |  Information for PDF file   |
%% -------------------------------
%% IM: Auto-Fill this information
\hypersetup{
 pdfauthor={\myname},
 pdftitle={\mytitle},
 pdfsubject={\mytype},
 pdfkeywords={\mytype}
}

%% ---------------------------------
%% | ToDo Marker - only for draft! |
%% ---------------------------------
% Remove this section for final version!
\setlength{\marginparwidth}{20mm}

%\newcommand{\margtodo}
%{\marginpar{\textbf{\textcolor{red}{ToDo}}}{}}

%\newcommand{\todo}[1]
%{{\textbf{\textcolor{red}{(\margtodo{}#1)}}}{}}


%% --------------------------------
%% | Old Marker - only for draft! |
%% --------------------------------
% Remove this section for final version!
\newenvironment{deprecated}
{\begin{color}{gray}}
{\end{color}}


%% --------------------------------
%% | Settings for word separation |
%% --------------------------------
% Help for separation:
% In german package the following hints are additionally available:
% "- = Additional separation
% "| = Suppress ligation and possible separation (e.g. Schaf"|fell)
% "~ = Hyphenation without separation (e.g. bergauf und "~ab)
% "= = Hyphenation with separation before and after
% "" = Separation without a hyphenation (e.g. und/""oder)

% Describe separation hints here:
\hyphenation{
% Pro-to-koll-in-stan-zen
% Ma-na-ge-ment  Netz-werk-ele-men-ten
% Netz-werk Netz-werk-re-ser-vie-rung
% Netz-werk-adap-ter Fein-ju-stier-ung
% Da-ten-strom-spe-zi-fi-ka-tion Pa-ket-rumpf
% Kon-troll-in-stanz
Lauf-zeit-ver-bes-se-run-gen
}


%% ------------------------
%% |    Including files   |
%% ------------------------
% Only files listed here will be included!
% Userful command for partially translating the document (for bug-fixing e.g.)
\includeonly{%
titlepage,
abstract,
text/abbreviations,
text/introduction,
text/content,
text/evaluation,
text/conclusion,
text/declaration,
text/appendix
}

%%%%%%%%%%%%%%%%%%%%%%%%%%%%%%%%%
%% Here, main documents begins %%
%%%%%%%%%%%%%%%%%%%%%%%%%%%%%%%%%
\begin{document}

\frontmatter
\pagenumbering{roman}
\include{titlepage}
\chapter{Acknowledgements}


\chapter{Abstract}

Closeness is a widely-studied centrality measure in network analysis. Computing the closeness centrality for all nodes in a graph does not scale to large real-world networks because it requires solving the all-pairs-shortest-path problem. For some applications, it is only necessary to find the $k$ most central nodes of a network. Prior work has shown that the Top-$k$ closeness centrality problem can be solved significantly faster on real-world networks than computing the closeness centrality for all nodes in the graph. In this thesis, we expand on this work and propose algorithms for Top-$k$ closeness centrality in dynamic graphs. Instead of recomputing the most central nodes from scratch after each modification of the graph, we use information obtained during earlier computations to update the list of the $k$ most central nodes faster. Our dynamic algorithms handle edge insertions and edge removals in directed and undirected unweighted graphs. We propose separate algorithms for complex social networks and street networks. We achieve speedups of up to three orders of magnitude compared to the static algorithm.

We also study the problem of finding a \emph{group} of $k$ nodes with high closeness. We adapt a static greedy algorithm for the dynamic case. We use our dynamic algorithm for Top-$k$ closeness centrality and employ similar techniques to reduce the number of recomputations for unaffected nodes. On average, our dynamic algorithm is between 8 and 15 times faster, depending on the group size.


\vspace{3cm}

\begingroup
\renewcommand{\cleardoublepage}{}
\renewcommand{\clearpage}{}
\chapter{Zusammenfassung}\label{chap:abstract_german}

\endgroup
% IM Style: No additional blank page
% \blankpage


%% -------------------
%% |   Directories   |
%% -------------------
\tableofcontents
% IM Style: No additional blank page
% \blankpaage

% Do not include a list of figures, list of tables and list of abbreviations,if the work is a seminar thesis
\iflanguage{english}{
\ifthenelse{\equal{\mytype}{Seminar Thesis}}
{}
{
\listoffigures \addcontentsline{toc}{chapter}{List of Figures} 
\listoftables  \addcontentsline{toc}{chapter}{List of Tables} 
%\printnomenclature   \addcontentsline{toc}{chapter}{List of Abbreviations} 
}
}
{
\ifthenelse{\equal{\mytype}{Seminararbeit}}
{}
{
\listoffigures \addcontentsline{toc}{chapter}{Abbildungsverzeichnis} 
\listoftables \addcontentsline{toc}{chapter}{Tabellenverzeichnis} 
%\printnomenclature \addcontentsline{toc}{chapter}{Abkürzungsverzeichnis}
}
}

% Execute this command for index creation, i.e., for abbreviations by the nomencl package

%makeindex thesis.nlo -s nomencl.ist -o thesis.nls



%\printnomenclature



%% -----------------
%% |   Main part   |
%% -----------------
\mainmatter
\pagenumbering{arabic}
%% introduction.tex
%%

%% ==============================
\chapter{Introduction}
\label{ch:Introduction}
%% ==============================

In network analysis, one fundamental problem is finding the most important and central nodes. Several measures have been proposed to ascertain the importance of nodes within their graphs. These measures can be used to find the most important people in social networks or central places in street networks.

% TODO: Cite Dijkstra
Closeness centrality is based on the intuition that a node is important if its distance to other nodes in the graph is small. Computing the closeness centrality of a node in an unweighted graph requires a complete breadth-first search (BFS), and a complete run of Dijkstra's algorithm on weighted graphs. In order to compute the closeness centrality of all nodes in the graph, one such search is required from all nodes. This is too expensive, especially on large real-world networks. If the application requires a complete ranking of all nodes, this effort can not be avoided.

For some applications, however, it is enough to compute a list of the $k$ most central nodes. This problem is called \emph{Top-k closeness}.



\paragraph{State of the Art}

\paragraph{Contributions}
\



%% content.tex
%%

%% ==============
%\input{text/chapters/definitions}

%% conclusion.tex
%%

%% ==================
\chapter{Conclusion}
\label{ch:conclusion}
%% ==================

We have developed and implemented dynamic algorithms for Top-$k$ closeness centrality for both complex networks and street networks. We initially limit the recomputation of closeness centralities or upper bounds thereof to nodes actually affected by an edge modification. Additionally, we can exploit the fact that we only need to compute the exact closeness centralities for the $k$ most central nodes. We can keep the upper bounds of nodes which are far away from an edge insertion in place and can update the upper bounds of so-called boundary nodes cheaply. At last, we can compute the maximum improvement of the closeness centrality for each node. If the resulting upper bound is smaller than the $k$-th largest closeness centrality, we can use the improvement bound and do not need a new breadth-first search.

Our dynamic algorithm is between 35 and 45 times faster on average than the static algorithm for edge insertions on undirected complex networks. On directed complex networks, the dynamic algorithm is between 15 and 20 times faster. In the best case, our dynamic algorithm is up to 700 times faster for some edges.

We achieve even higher speedups on street networks, improving by a factor of more than 100 on average with directed street networks. In some cases, our dynamic algorithm is up to three orders of magnitudes faster than the static algorithm. Our analysis shows that the speedup is the result of a much lower number of complete breadth-first searches compared to the static algorithm.

We have also implemented an incremental algorithm based on the greedy algorithm for the Maximum Closeness Centrality Group Identification problem. We make use of our dynamic algorithm for Top-$k$ closeness centrality and try to replicate the results of the static algorithm with as little computational effort as possible. We avoid recomputing the marginal gains for as many nodes as possible. Once any node in the group has changed, we fall back to the static algorithm.

The dynamic algorithm is between 8 and 15 times faster than the static algorithm on average, and up to 800 times faster for some inserted edges on the larger networks we have tested.

\paragraph{Future work}
We have already implemented a simple optimization to speed up the computation of the number of reachable nodes for each node in undirected graphs after an edge insertion. As we have seen in our experimental evaluation, recomputing an upper bound for the number of reachable nodes in directed graphs after edge modifications is a significant bottleneck. There are algorithms that maintain reachability information in directed graphs~\cite{lkacki2013improved,chechik2016decremental} after edge removals. Since these algorithms maintain exact reachability information for single source nodes, it might be impractical to use them for our purposes. However, it might be practical to maintain upper bounds for the number of reachable nodes.

Our dynamic algorithm is currently designed to update the ranking of the $k$ most central nodes after every single edge modification. However, it might not be necessary to update the Top-$k$ nodes after each edge modification. Then it might be possible to coalesce multiple edge insertions or edge removals into a single update operation and still be faster than running the static algorithm. This is called \emph{batch update}. One could compute the set of affected nodes individually for each edge in the batch and compute the union of all these sets. Even far-away nodes could still be skipped for edge insertions if their minimum distance to any of the edges is larger than their previous cutoff level.

In our dynamic algorithm for group closeness, we perform a pruned BFS for each affected node in each iteration of the algorithm. It might be possible to adapt the idea of the level-based improvement bounds for the group closeness algorithm. If it is possible to compute an upper bound for the increase of the marginal gain for all nodes, we could skip the expensive pruned BFS for nodes where we are sure that the new marginal gain will be smaller than the currently known maximum.



\include{text/declaration}


%% ----------------
%% |   Appendix   |
%% ----------------
% IM Style: No additional blank page
% \cleardoublepage

%\input{text/appendix}


%% --------------------
%% |   Bibliography   |
%% --------------------
\cleardoublepage
\phantomsection
\addcontentsline{toc}{chapter}{\bibname}

% IM Style
\iflanguage{english}
{\bibliographystyle{plain}}	% english style
{\bibliographystyle{plain}}	% german style

%

% Informatik-Style
%\iflanguage{english}
%{\bibliographystyle{IEEEtranSA}}	% english style
%{\bibliographystyle{babalpha-fl}}	% german style
												  
% Use IEEEtran for numeric references
%\bibliographystyle{IEEEtranSA})

\bibliography{thesis}
\listoftodos

\end{document}
